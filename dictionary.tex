%%%%%%%%%%%%%%%%%%%%%%%%%%%%%%%%%%%%%%%%%
% Dictionary
% LaTeX Template
% Version 1.1 (6/8/17)
%
% This template was downloaded from:
% http://www.LaTeXTemplates.com
%
% Original author:
% Vel (vel@latextemplates.com) inspired by a template by Marc Lavaud
%
% License:
% CC BY-NC-SA 3.0 (http://creativecommons.org/licenses/by-nc-sa/3.0/)
%
%%%%%%%%%%%%%%%%%%%%%%%%%%%%%%%%%%%%%%%%%

%----------------------------------------------------------------------------------------
%	PACKAGES AND OTHER DOCUMENT CONFIGURATIONS
%----------------------------------------------------------------------------------------

\documentclass[10pt,letter,twoside]{article} % 10pt font size, A4 paper and two-sided margins
\usepackage{adjustbox}
\usepackage[top=3.5cm,bottom=3.5cm,left=3.7cm,right=4.7cm,columnsep=30pt]{geometry} % Document margins and spacings
\usepackage[utf8]{inputenc} % Required for inputting international characters
\usepackage[T1]{fontenc} % Output font encoding for international characters
\usepackage{color}
\usepackage{palatino} % Use the Palatino font
\usepackage{wallpaper}
\usepackage{tabularx}
\usepackage{makecell}%To keep spacing of text in tables
\usepackage{booktabs}
\usepackage{multirow}

\setcellgapes{5pt}%parameter for the spacing
\newcolumntype{L}{>{\raggedright\arraybackslash}X}
\usepackage{float}
\ULCornerWallPaper{1}{fondo.pdf}
\usepackage{microtype} % Improves spacing

\usepackage{multicol} % Required for splitting text into multiple columns

\usepackage[bf,sf,center]{titlesec} % Required for modifying section titles - bold, sans-serif, centered

\usepackage{fancyhdr} % Required for modifying headers and footers
\fancyhead[L]{\textsf{\rightmark}} % Top left header
\fancyhead[R]{\textsf{\leftmark}} % Top right header
\renewcommand{\headrulewidth}{1.4pt} % Rule under the header
\fancyfoot[R]{\vspace{12mm}\textbf{\textcolor{white}{\textsf{LATINO BI | \thepage}}}} % Bottom center footer
%\renewcommand{\footrulewidth}{1.4pt} % Rule under the footer
\pagestyle{fancy} % Use the custom headers and footers throughout the document

\newcommand{\entry}[4]{\textbf{#1}\markboth{#1}{#1}\ {(#2)}\ \textit{#3}\ $\bullet$\ {#4}} % Defines the command to print each word on the page, \markboth{}{} prints the first word on the page in the top left header and the last word in the top right
\renewcommand{\familydefault}{\sfdefault}
\setlength{\parindent}{0pt}
\setlength{\fboxrule}{1mm}
\setlength{\fboxsep}{6mm}
\usepackage{tikz}
\usepackage{amsmath}
\usepackage{amsmath,amssymb}%      pour les maths
\usepackage{enumitem}
\usepackage{varwidth}
\definecolor{blued}{rgb}{0.122, 0.435, 0.698}

\usetikzlibrary{calc}

\newcommand{\mybox}[4][\textwidth-\pgfkeysvalueof{/pgf/inner xsep}-2mm]{%
	\begin{figure}[!h]
		\centering
		\begin{tikzpicture}
		\node[line width=.5mm, rounded corners, draw=#2, inner ysep=10pt, text width=#1, outer sep=0] (one) {\vspace*{15pt}\\\begin{varwidth}{\textwidth}#4\end{varwidth}};
		\node[text=white,anchor=north east,align=center, minimum height=20pt] (two) at (one.north east) {#3 \hspace*{.5mm}};
		\path[fill=#2] 
		(one.north west|-two.west) --
		($(two.west)+(-1.5cm,0)$) 
		to[out=0,in=180] (two.south west) --
		(two.south east) [rounded corners] --
		(one.north east) -- 
		(one.north west) [sharp corners] -- cycle;
		\node[text=white,anchor=north east,align=center, minimum height=20pt, text height=2ex] (three) at (one.north east) {#3 \hspace*{.5mm}};
		\end{tikzpicture} 
	\end{figure}
}


%----------------------------------------------------------------------------------------

\begin{document}

%----------------------------------------------------------------------------------------
%	SECTION A
%----------------------------------------------------------------------------------------


\section*{DICCIONARIO DE MODELOS }



\mybox{blued!70!black}{¿Cómo leer este Documento?}{
	Este documento contiene tres secciones importantes:
	\begin{enumerate}
		\item \textbf{Convenciones:} Representa la nomenclatura usada para las métricas.\\ Esta sección presenta \textbf{Nombre Convención} (Tabla asociada a la convención) $\bullet$ Descripción.
		
		\item \textbf{Movimientos:} Describe la funcionalidad de las tablas de transacciones, asociadas al Modelo Comercial.\\ Esta sección presenta \textbf{Nombre de la Tabla} (Tipo de tabla (Búsqueda, Movimiento)) $\bullet$ Descripción.
		
		\item \textbf{Métricas:} Describe la funcionalidad de la métrica dentro del contexto del negocio.\\ Esta sección presenta \textbf{Nombre de la Métrica} (Cubos en los que se encuentra la métrica) \textit{Tabla dueña de la métrica} $\bullet$ Descripción y fórmula.
\end{enumerate}}


\section*{MODELO COMERCIAL}
\begin{multicols}{2}

\section*{CONVENCIONES}

\entry{DHK}{Despachos HongKong}{}{Todas las métricas asociadas a los Despachos HongKong, se identificaran  con (DHK)}\\

\entry{PV}{Pedido de Venta}{}{Todas las métricas asociadas a los pedidos de venta, se identifican con (PV)}\\

\entry{PC}{Presupuesto Comercial}{}{Todas las métricas asociadas al presupuesto comercial, se identifican con (PC)}\\

\entry{VC}{Ventas Compañía}{}{Todas las métricas asociadas a las Ventas de la Compañía, se identifican con (PC)}\\

\section*{MOVIMIENTOS}

Los movimientos representan los procesos de negocio de la compañía, por tal motivo son la base para el análisis.\\ 

\entry{M Despachos HongKong}{Movimiento}{ }{Descripción de la Fact}\\

\entry{M Pedido de Venta}{Movimiento}{}{Descripción de la Fact}\\

\entry{PC}{Presupuesto Comercial}{}{Descripción de la Fact}\\

\entry{VC}{Ventas Compañía}{}{Descripción de la Fact}\\



\newpage

\onecolumn
\section*{Tablas de Búsqueda}
Las tablas de búsqueda representan todas las posibles perspectivas desde las cuales puede ser analizadas las tablas de movimientos. 
\section*{Condición de Pago}
\begin{table}[H]
	\centering
	\makegapedcells
	\begin{tabularx}{\linewidth}{|L|c|L|} 
		\hline
		\multicolumn{1}{|c|}{Nombre Tabla}   & \multicolumn{1}{c|}{Nombre Campo}                                    & \multicolumn{1}{c|}{Descripción Campo}                                                                                                   \\ \hline
		D Condición de Pago & \begin{tabular}[c]{@{}l@{}}Código Condición\\   de Pago\end{tabular} & Representa el Código con el que se identifican de manera única las condiciones de pago  \vfill                                                 \\ \cline{2-3} 
		& Condición de Pago                                                    & Representa un acuerdo establecido con clientes y proveedores como tipos de descuento, plazos de pago, condiciones de entrega entre otros \vfill\\ \hline
	\end{tabularx}
\end{table}

\section*{Compañía}
\begin{table}[H]
	\centering
	\makegapedcells
	\begin{tabularx}{\linewidth}{|L|c|L|} 
		\hline
		\multicolumn{1}{|c|}{Nombre Tabla}   & \multicolumn{1}{c|}{Nombre Campo}                                    & \multicolumn{1}{c|}{Descripción Campo}                                                                                                   \\ \hline
		D Compañía & \begin{tabular}[c]{@{}l@{}}Código Compañía\end{tabular} & Representa el Código con el que se identifican de manera única las compañías. \vfill                                                 \\ \cline{2-3} 
		& Compañía                                                   & Representa la compañía (NASA, NALSANI). \vfill\\ \hline
	\end{tabularx}
\end{table}


\newpage
\section*{Moneda}
\begin{table}[H]
	\centering
	\makegapedcells
	\begin{tabularx}{\linewidth}{|L|c|L|} 
		\hline
		\multicolumn{1}{|c|}{Nombre Tabla}   & \multicolumn{1}{c|}{Nombre Campo}                                    & \multicolumn{1}{c|}{Descripción Campo}                                                                                                   \\ \hline
		D Moneda & \begin{tabular}[c]{@{}l@{}}Código Moneda\end{tabular} & Representa el Código con el que se identifican de manera única las Monedas. \vfill                                                 \\ \cline{2-3} 
		& Moneda                                                   & Unidad representativa del precio de las cosas que permite efectuar transacciones comerciales. \vfill\\ \hline
	\end{tabularx}
\end{table}



\section*{Tipo de Transacción}
\begin{table}[H]
	\centering
	\makegapedcells
	\begin{tabularx}{\linewidth}{|L|c|L|} 
		\hline
		\multicolumn{1}{|c|}{Nombre Tabla}   & \multicolumn{1}{c|}{Nombre del Campo}                                    & \multicolumn{1}{c|}{Descripción del Campo}                                                                                                   \\ \hline
		D Tipo de Transacción & \begin{tabular}[c]{@{}l@{}}Código Tipo de Transacción\end{tabular} & Representa el Código con el que se identifican de manera única los tipos de transacción. \vfill                                                 \\ \cline{2-3} 
		& Tipo de Transacción                                                   & NO APARECE EN SMARTSHEET \vfill\\ \hline
	\end{tabularx}
\end{table}

\section*{Tipo de Presupuesto}
\begin{table}[H]
	\centering
	\makegapedcells
	\begin{tabularx}{\linewidth}{|L|c|L|} 
	\hline
	Nombre de la Tabla                     & Nombre del Campo        & Descripción del Campo                                                                                           \\ \hline
	\multirow{2}{*}{D Tipo de Presupuesto} & Código Tipo Presupuesto & Representa el Código con el que se identifican de manera única los tipos de presupuesto que asigna la compañía. \\ \cline{2-3} 
	& Tipo Presupuesto        & Es la clasificación otorgada a los diferentes presupuesto que asigna la compañía.                               \\ \hline
	
	\end{tabularx}
\end{table}

\newpage
\section*{Centro de Costo}
\begin{table}[H]
	\centering
	\makegapedcells
	\begin{tabularx}{\linewidth}{|L|c|L|} 
		\hline
		Nombre de la Tabla               & Nombre del Campo & Descripción del Campo                                                                                                   \\ \hline
		\multirow{3}{*}{D Centro de Costo} & Centro de Costo  & NO APARECE EN SMARTSHEET                                                                                                \\ \cline{2-3} 
		& Canal            & Es la estructura de negocio encargada de llevar el producto hasta el cliente desde el punto de origen                   \\ \cline{2-3} 
		& Gerencia         & Persona o conjunto de personas que se encargan de dirigir, gestionar o administrar una sociedad, empresa u otra entidad \\ \hline
	\end{tabularx}
\end{table}

\section*{Geografía}
\begin{table}[H]
	\centering
	\makegapedcells
	\begin{tabularx}{\linewidth}{|L|c|L|} 
		\hline
		Nombre de la Tabla           & Nombre del Campo & Descripción del Campo              \\ \hline
		\multirow{5}{*}{D Geografía} & País             & Nombre del País                    \\ \cline{2-3} 
		& Departamento     & Nombre del Departamento            \\ \cline{2-3} 
		& Ciudad           & Nombre de la Ciudad                \\ \cline{2-3} 
		& Código Postal    & Número del Código Postal           \\ \cline{2-3} 
		& Población        & Conjunto de Habitantes de un lugar \\ \hline
	\end{tabularx}
\end{table}


\section*{Dirección Comercial}
\begin{table}[H]
	\centering
	\makegapedcells
	\begin{tabularx}{\linewidth}{|L|c|L|} 
		\hline
		Nombre de la Tabla    & Nombre del Campo & Descripción del Campo    \\ \hline
		D Dirección Comercial & Dirección        & NO APARECE EN SMARTSHHET \\ \hline
	\end{tabularx}
\end{table}

\newpage


\section*{Tiempo}
\begin{table}[H]
	\centering
	\makegapedcells
	\begin{tabularx}{\linewidth}{|L|c|L|} 
	\hline
	Nombre Tabla              & Nombre del Campo    & Descripción del Campo                                                                                    \\ \hline
	\multirow{9}{*}{D Tiempo} & Fecha               & Es una representación del tiempo en formato (DD/MM/AAAA)                                                 \\ \cline{2-3} 
	& Temporada           & Período caracterizado por algo o destinado a cierta actividad                                            \\ \cline{2-3} 
	& Año Comercial       & Representa el período de tiempo comprendido por 12 meses, iniciando en Abril y finalizando en Marzo.     \\ \cline{2-3} 
	& Semestre Comercial  & Representa el período de tiempo comprendido por 6 meses, iniciando en Abril y finalizando en Septiembre. \\ \cline{2-3} 
	& Trimestre Comercial & Representa el período de tiempo comprendido por 3 meses, iniciando en Abril y finalizando en Junio.      \\ \cline{2-3} 
	& .Mes Comercial      & Representa un período de tiempo generalmente comprendido por 30 días.                                    \\ \cline{2-3} 
	& Semana Comercial    & Representa un período de tiempo comprendido por 7 días.                                                  \\ \cline{2-3} 
	& Día de la Semana    & Representa el nombre del día de la semana.                                                               \\ \cline{2-3} 
	& Día                 & Representa un intervalo de tiempo comprendido por 24 horas, enuncia el nombre del mes y su día.          \\ \hline
	\end{tabularx}
\end{table}

\newpage
\section*{Cliente}
\begin{table}[H]
	\centering
	\makegapedcells
	\begin{adjustbox}{max height=0.5\textheight}
		\begin{tabularx}{\linewidth}{|L|c|L|} 
			\hline
			Nombre de la Tabla          & Nombre del Campo         & Descripción del Campo                                                                                         \\ \hline
			\multirow{14}{*}{D Cliente} & Cliente                  & Nombre de la persona que adquiere un producto o servicio                                                      \\ \cline{2-3} 
			& Latitud                  & Representa un coordenada de georeferenciación                                                                 \\ \cline{2-3} 
			& Longitud                 & Representa un coordenada de georeferenciación                                                                 \\ \cline{2-3} 
			& Género                   & Conjuntode personas o cosas que tienen características generales comunes                                      \\ \cline{2-3} 
			& Límite de Crédito        & NO APARECE EN SMARTSHHET                                                                                      \\ \cline{2-3} 
			& Número de Identificación & Número único que permite identificar ante entidades estatales a una persona u organización (Cédula, NIT, etc) \\ \cline{2-3} 
			& Cuenta del Cliente       & NO APARECE EN SMARTSHHET                                                                                      \\ \cline{2-3} 
			& Grupo del Cliente        & NO APARECE EN SMARTSHHET                                                                                      \\ \cline{2-3} 
			& País                     & Nombre del País en el que se encuentra el cliente                                                             \\ \cline{2-3} 
			& Depatamento              & Nombre del Departamento en el que se encuentra ubicado el cliente                                             \\ \cline{2-3} 
			& Ciudad                   & Nombre de la Ciudad en la que se encuentra ubicado el cliente                                                 \\ \cline{2-3} 
			& Código Sucursal          & Código que representa de manera única cada una de las sucursales de la compañía.                              \\ \cline{2-3} 
			& Sucrusal                 & Nombre del establecimiento que depende de la compañía                                                         \\ \cline{2-3} 
			& Dirección                & Indicación precisa del lugar en el que se encuentra el cliente   \\ \hline 
		\end{tabularx}
	\end{adjustbox}
	
\end{table}



\twocolumn
\section*{MÉTRICAS}

\entry{Unidades Despachadas (DHK)}{Cubo Comercial-Cubo Ventas POS}{M  Despachos HongKong}{NO HAY DEFINICIÓN EN SMART SHEET.	
 \underline{\textbf{Fórmula:}} SUM('M  Despachos HongKong'[Unidades Despachadas])}\\


\entry{Valor Despacho COP (DHK)}{Cubo Comercial-Cubo Ventas POS}{M  Despachos HongKong}{NO HAY DEFINICIÓN EN SMART SHEET.	
	\underline{\textbf{Fórmula:}} SUM('M  Despachos HongKong'[Valor Despacho COP])}\\

\entry{Valor Despacho USD (DHK)}{Cubo Comercial-Cubo Ventas POS}{M  Despachos HongKong}{NO HAY DEFINICIÓN EN SMART SHEET.	
	\underline{\textbf{Fórmula:}} SUM('M  Despachos HongKong'[Valor Despacho USD])}\\


\entry{\% Unidades Facturadas / Reservadas (PV)}{Cubo Comercial-Cubo Ventas POS}{M  Pedido de Venta}{Porcentaje de comparación entre las Unidades Facturadas	
	\underline{\textbf{Fórmula:}} SUM('M  Despachos HongKong'[Valor Despacho USD])}\\


\entry{Aberration}{ab-uh-rey-shuh n}{Noun}{The act of deviating from the ordinary, usual, or normal type.}

\entry{Above}{uh-buhv}{Preposition}{In extended space over and not touching.}

\entry{Academia}{ak-uh-dee-mee-uh}{Noun}{The environment or community concerned with the pursuit of research, education, and scholarship.}

\entry{Accomplished}{uh-kom-plisht}{Adjective}{Completed; done; effected. Highly trained or skilled in a particular activity.}

\entry{Acidophilic}{uh-sid-uh-fil-ik, as-i-duh-}{Adjective}{Biology: having an affinity for acid stains; eosinophilic. Ecology: thriving in or requiring an acid environment.}

\entry{Adaptation}{ad-uh p-tey-shuh n}{Noun}{The action or process of adapting or being adapted. Biology: The process of change by which an organism or species becomes better suited to its environment}

\entry{Adenine}{ad-n-in, -een, -ahyn}{Noun}{A compound which is one of the four constituent bases of nucleic acids. A purine derivative, it is paired with thymine in double-stranded DNA.}

\entry{Adorable}{uh-dawr-uh-buh l}{Adjective}{Inspiring great affection or delight.}

\entry{Advanced}{ad-vanst}{Adjective}{Far on or ahead in development or progress.}

\entry{Aerial}{air-ee-uh l}{Noun}{A rod, wire, or other structure by which signals are transmitted or received as part of a radio or television transmission or receiving system.}

\entry{Affordable}{uh-fawr-duh-buh l}{Adjective}{Believed to be within one's financial means.}

\entry{Agnostic}{ag-nos-tik}{Noun}{A person who holds that the existence of the ultimate cause, as God, and the essential nature of things are unknown and unknowable, or that human knowledge is limited to experience.}

\entry{Aioli}{ahy-oh-lee}{Noun}{Mayonnaise seasoned with garlic.}

\entry{Alchemy}{al-kuh-mee}{Noun}{The medieval forerunner of chemistry, concerned with the transmutation of matter, in particular with attempts to convert base metals into gold or find a universal elixir.}

\entry{Algebra}{al-juh-bruh}{Noun}{The part of mathematics in which letters and other general symbols are used to represent numbers and quantities in formulae and equations. }

\entry{Amatol}{am-uh-tawl}{Noun}{A high explosive consisting of a mixture of TNT and ammonium nitrate.}

\entry{Almanac}{awl-muh-nak}{Noun}{An annual publication containing a calendar for the coming year, the times of such events and phenomena}

\entry{Animal}{an-uh-muh l}{Noun}{A living organism which feeds on organic matter, typically having specialized sense organs and nervous system and able to respond rapidly to stimuli.}

\entry{Ascension}{auh-sen-shuh n}{Noun}{The action of rising to an important position or a higher level.}

\entry{Aspire}{uh-spahyuh r}{Verb}{Direct one's hopes or ambitions towards achieving something.}

\entry{Athlete}{ath-leet}{Noun}{a person trained or gifted in exercises or contests involving physical agility, stamina, or strength; a participant in a sport, exercise, or game requiring physical skill.}

\entry{Azobenzene}{az-oh-ben-zeen}{Noun}{A synthetic crystalline organic compound used chiefly in dye manufacture.}

\end{multicols}

%----------------------------------------------------------------------------------------
%	SECTION B
%----------------------------------------------------------------------------------------

\section*{MODELO FINANCIERO}

\begin{multicols}{2}

\entry{Babble}{bab-uh l}{Verb}{Talk rapidly and continuously in a foolish, excited, or incomprehensible way.}

\entry{Balance}{bal-uh ns}{Noun}{An even distribution of weight enabling someone or something to remain upright and steady. An instrument for determining weight, typically by the equilibrium of a bar with a fulcrum at the center, from each end of which is suspended a scale or pan, one holding an object of known weight, and the other holding the object to be weighed.}

\entry{Barbet}{bahr-bit}{Noun}{A large-headed, brightly coloured fruit-eating bird that has a stout bill with tufts of bristles at the base. Barbets are found on all continents, especially in the tropics.}

\entry{Beetroot}{beet-root}{Noun}{The edible dark-red spherical root of a kind of beet, eaten as a vegetable.}

\entry{Besides}{bih-sahydz}{Preposition}{In addition to; apart from.}

\entry{Bevel}{bev-uh l}{Noun}{A slope from the horizontal or vertical in carpentry and stonework; a sloping surface or edge.}

\entry{Bevel}{bev-uh l}{Noun}{A slope from the horizontal or vertical in carpentry and stonework; a sloping surface or edge.}

\entry{Biennial}{bahy-en-ee-uh l}{Adjective}{Taking place every other year.}

\entry{Bioinformatics}{bahy-oh-in-fer-mat-iks}{Noun}{The retrieval and analysis of biochemical and biological data using mathematics and computer science, as in the study of genomes. The science of collecting and analysing complex biological data such as genetic codes.}

\entry{Bleep}{bleep}{Noun}{A short high-pitched sound made by an electronic device as a signal or to attract attention.}

\entry{Blind}{blahynd}{Adjective}{Unable to see; lacking the sense of sight; sightless.}

\entry{Bonanza}{buh-nan-zuh}{Noun}{A situation which creates a sudden increase in wealth, good fortune, or profits.}

\entry{Book}{boo k}{Noun}{A written or printed work consisting of pages glued or sewn together along one side and bound in covers.}

\entry{Bran}{bran}{Noun}{The partly ground husk of wheat or other grain, separated from flour meal by sifting.}

\entry{Break}{breyk}{Verb}{Separate into pieces as a result of a blow, shock, or strain.}

\entry{Bridge}{brij}{Noun}{A structure carrying a road, path, railway, etc. across a river, road, or other obstacle. Music: The part of a stringed instrument over which the strings are stretched.}

\entry{Brioche}{bree-ohsh}{Noun}{A light sweet yeast bread typically in the form of a small round roll.}

\entry{Buzzard}{buhz-erd}{Noun}{A large hawklike bird of prey with broad wings and a rounded tail, often seen soaring in wide circles.}

\entry{Bystander}{bahy-stan-der}{Noun}{A person who is present at an event or incident but does not take part.}

\end{multicols}

%------------------------------------------------
\end{document}